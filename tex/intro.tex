\chapter{Introduction}
CloudSat and CALIPSO are Earth observation satellites launched and operated by
NASA, and NASA
and CNES (respectively), developed in order to deepen our understanding of
clouds,
aerosols, and how they influence the Earth's radiation budget. As of June 2010
they are still in operation, and so far they have generated hundreds of
terabytes of data, ranging from raw sensor measurement to highly processed
measurements combined with data from other satellites, model outputs and
databases. All this information can only be useful to the broader scientific
community if they are provided with tools that enable them to visualise it.
Although the CloudSat and CALIPSO missions provide such tools, they have a
number of shortcomings, which make them unsuitable in some situations. They are
either in the form of \program{ITTVIS~IDL} and \program{MATLAB} scripts, which
are expensive proprietary platforms, or provided through the web service
\program{Giovanni}, which does not offer the flexibility of a standalone
application. There is also an implementation of a CALIPSO visualisation tool for
\program{Google~Earth}, but it does not provide the level of details of
traditional two-dimensional forms of presentation and publication quality
output.
To accommodate the needs of our research at the \textit{Department of Astronomy,
Physics of the Earth and Meteorology} of the \textit{Comenius University}, it
was necessary to develop a computer program which would allow us to visualise 
this data efficiently and in high quality.
The aim of this thesis was to develop and document such a tool. We shall call it
\ccplot.

\section{Goals}
The following features were desired:
\begin{itemize}
\item \textbf{Products support.} Support several data sets, especially:
\begin{itemize}
\item \textbf{CALIPSO:} Total Attenuated Backscatter \SI{532}{nm}, Attenuated Backscatter \SI{1064}{nm},
Perpendicular Attenuated Backscatter \SI{532}{nm}, Integrated Attenuated Backscatter \SI{532}{nm},
Integrated Attenuated Backscatter \SI{1064}{nm}
\item \textbf{CloudSat:} Radar Reflectivity Factor
\item \textbf{MODIS Level 1:} Band 1 RSB, Band 31 TEB
\end{itemize}
\item \textbf{Extent selection.} Allow plot extent to be selected
vertically by height, and horizontally by rays, time, geographical coordinates,
scanlines and samples.
\item \textbf{Color maps.} Custom color maps for the supported data sets should be constructed.
\item \textbf{Map projections.} At least Transverse Mercator projection should be supported.
\item \textbf{Interpolation algorithm.} The interpolation algorithm should preserve
as much details as possible.
\item \textbf{Scriptability.} The program should provide interface which allows
it to be used with other programs and scripts.
\end{itemize}


\section{CloudSat and CALIPSO Satellites}

The CloudSat and CALIPSO missions are build on the legacy of ground-based
rainfall radars and airborne lidars, but the extent of their coverage is
unmatched by any ground or air operations. CloudSat is a satellite equipped
with a millimetre-wavelength radar especially suited for studying the properties
of warm and mixed-phase clouds. The primary products coming from CloudSat are
two-dimensional vertical cross-sections of the atmosphere extending from the
ground to some \SI{30}{km} of altitude.

CALIPSO is equipped with a visible and near-infrared lidar. Due to its small
wavelength, it is capable of detecting ice-phase clouds and aerosols. The primary
products are cross-sections of the atmosphere similar to those of CloudSat.
Because CloudSat and CALIPSO fly in a formation no more than \SI{15}{s} apart, their
output can be readily matched to make a more comprehensive view of the
situation.

Thanks to the global scale and time continuity of the measurements CloudSat
and CALIPSO provide a great potential for advancement of our understanding of
processes influencing the Earth's climate system. The data can also aid
meteorologists in studying the structure and development of clouds and aerosols,
although not in weather forecasting, because of the intermittent character of
coverage inherent to polar-orbiting satellites.

Data from CloudSat and CALIPSO are well-suited to be combined with
measurements from the visible and infrared spectroradiometer MODIS carried on
board of
the satellite Aqua, thereby providing the user with information about broader
circumstances surrounding the situation that is being studied.

CloudSat, CALIPSO and Aqua are members of a larger formation of Earth
observation satellites \textit{A-Train},
operated mostly by NASA.

\section*{Organisation}
\begin{itemize}
\item\textbf{Chapter 1. Introduction} explains what
the CloudSat and CALIPSO satellites are, and what kind of measurements they
provide.

\item\textbf{Chapter 2. A-Train} starts by introducing the A-Train constellation
of
satellites, and continues by
describing their instruments and data products provided by the missions.

\item\textbf{Chapter 3. The Physical Basis} aims to present the reader with the
scientific
basis underlying the operation of radars and lidars by first introducing the
physical laws governing radiation transfer and scattering in the atmosphere, and
secondly by explaining equations used by the missions in converting of raw data
originating from detectors into useful physical quantities suitable for
visualisation.

\item\textbf{Chapter 4. HDF4 Data Files} provides information on the technical
details of the
file standard employed by NASA EOS missions for storage and distribution of
their products --- the HDF4 and HDF-EOS2 file formats. Particularities of
CloudSat, CALIPSO and MODIS such as the file naming conventions and fine data
structure are also discussed in this chapter.

\item\textbf{Chapter 5. Visualising CloudSat} and CALIPSO Data strives to
document the design
and implementation details of \ccplot, including the relationship of various
components of \ccplot, calculations of plot dimensions, and the interpolation
and
layer processing algorithms.

\item\textbf{Chapter 6. ccplot Manual} is a standalone part of the thesis aimed
at proving a
first-time \ccplot user with an easy to follow tutorial, diving into more
complex
aspects such as plot customisation and description how to make your own color
maps.

\item\textbf{Chapter 7. Conclusion} summarises the outcomes of our work on
\ccplot, and
presents visions we have for the future versions of \ccplot.

\item\textbf{Appendix A. ccplot Manual Page} is a full \ccplot reference in the
form of a traditional
UNIX manual
page. 
\item\textbf{Appendix B. Case Studies} contains three case studies serving as a
visual presentation of
what can be achieved with \ccplot.
\end{itemize}


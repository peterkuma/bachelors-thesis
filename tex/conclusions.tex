\chapter{Conclusion}
As stated in the \textit{Introduction}, the aim of this thesis was to develop a tool called \ccplot
which would allow us to visualise data originating from CloudSat and CALIPSO.
We developed and released this program under the open-source-compatible, two-clause BSD license. As such,
\ccplot and its source code is available to any one with no charge, and
it can be modified and redistributed without restrictions. The third-party tools
necessary to run \ccplot are released under similar permissive licenses. It can
be run on a number of unix-compatible operating system platforms including
\program{Linux}. \ccplot can be downloaded from \url{http://www.ccplot.org}.
Presentation of three case studies of \ccplot can be found in Appendix~B.

The development of \ccplot started with a simple goal of being able to
visualise a few data sets from CloudSat, CALIPSO and MODIS products. Over the course of several months it grew
from a few lines of code of the very first implementation into a much more
complex program of more than 2700 lines. It is now able to handle more
sophisticated extent selection, plot layout customisation, multiple projections
and custom color maps. We started a web site to make it available to any one who
needs it. By opening it as an open source project, we hope to
attract new developers, encourage users to help with reporting and fixing of
flaws in the program, and providing them with a place for discussion.

However, during the development we encountered many areas where improvement can
be made. To mention just a few, some of the libraries can be replaced with
alternatives that will provide greater performance and lower-level access, such as
the \program{cairo} 2D drawing library or \program{proj.4} map projection toolkit. More
consistent application models for holding data structures can be introduced, so
that support for new products can be added more readily.
There are plans to build a GUI version to make the experience of working with \ccplot more
pleasurable to the user. Because there are many approaches to choose from, we
want decide according to feedback from users of the current version of
\ccplot. For example, it can be designed to allow for a very detailed examination
of a single product file, or it can be designed to combine many products effectively,
thereby enabling the user to handle large amounts of data quickly and comprehensively.
Last but not least, we would like \ccplot to become a universal tool capable
of visualising most scientific data coming from CloudSat and CALIPSO (and
potentially other A-Train members as well), but that is a goal yet to be
achieved.

\def\abstractname{Abstrakt}
\begin{abstract}
\textbf{%
Kuma, P., 2010: Vizualizácia dát zo satelitov CloudSat a CALIPSO.
Univerzita Komenského, Bratislava, Fakulta matematiky, fyziky a informatiky,
Katedra astronómie, fyziky Zeme a meteorológie. 77pp. Školiteľ: Bašták Ďurán, I.}\\

\noindent CloudSat a CALIPSO sú pozemské satelity obiehajúce na polárnej dráhe
prevádzkované NASA, resp. NASA a CNES. CloudSat nesie radar vysielajúci na milimetrovej
vlnovej dĺžke na pozorovanie oblakov. CALIPSO nesie polarizačný lidar merajúci
vo viditeľnom a infračervenom spektre na pozorovanie aerosólov a studených oblakov.
Dáta z týchto satelitov sú distribuované vo forme HDF4 a \mbox{HDF-EOS2} súborov.
Predstavujeme softwarový nástroj \textit{ccplot} schopný vizualizovať niekoľko dátových
zostáv z produktov \textit{CloudSat 2B-GEOPROF},
\textit{CALIPSO Lidar L1B Profiles Products}, \textit{CALIPSO Lidar L2 Cloud Layer Products}
a \textit{Aqua MODIS L1B Products}. \textit{ccplot} je skriptovateľný, unixový nástroj
pracujúci cez príkazový riadok. \textit{ccplot} sme uvoľnili na Internete pod open-source kompatibilnou BSD licenciou.\\

\noindent\textbf{Kĺúčové slová:} CloudSat, CALIPSO, MODIS, A-Train, HDF4, HDF-EOS2, ccplot, graf,
vizualizácia
\end{abstract}
\def\abstractname{Abstract}
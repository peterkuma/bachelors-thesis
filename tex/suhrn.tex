


\chapter*{Vizualizácia dát zo satelitov CloudSat a CALIPSO}
\section*{\LARGE Súhrn práce}
\chaptermark{Súhrn}
CloudSat a CALIPSO sú satelity obiehajúce okolo Zeme na polárnej dráhe, slúžiace
primárne na meranie oblakov a aerosólov. Prevádzkované sú organizáciami NASA a
CNES. 
Sú súčasťou formácie satelitov
nazývanej A-Train (Afternoon Train, čiže poobedný), lebo prelietajú cez rovník
približne o 13:30 stredného miestneho času. Vďaka svojej blízkosti (iba asi
\SI{15}{min}) a rozmanitosti meracích prístrojov poskytujú bohaté informácie o
atmosfére a povrchu Zeme na danom mieste a v danom čase.

CloudSat nesie na palube radar CPR podobný pozemným zrážkovým rádiolokátorom
využívanými meteorologickými službami, pracujúci na frekvencii \SI{94}{GHz}. Slúži na
vytvára\-nie dvojrozmerných rezov atmosférou od povrchu do \SI{30}{km}, na ktorých sú
identifikovateľné hlavne oblaky tvorené z kvapiek.

CALIPSO nesie lidar (laserový radar) CALIOP, pracujúci vo viditeľnom a
infračervenom spektre (presnejšie vlnová dĺžka \SI{532}{nm}, resp. \SI{1064}{nm}). Jeho hlavné
produky sú podobné ako produkty CPR, čiže vertikálne rezy atmosférou. Vďaka
kratšej vlnovej dĺžke je citlivý na menšie objekty, hlavne aerosóly a oblaky
tvorené z ľadových kryštálov.

Satelity CloudSat a CALIPSO, ale aj ostatné členy zostavy A-Train, sú vďaka
širokej plejáde produktov, globálnemu pokrytiu a súbežnosti meraní veľmi
užitočným nástrojom pre výskum atmosféry a klimatického systému Zeme. Sú
využívané tak v meteorologickom, ako aj klimatologickom výskume.

Keďže dostupné aplikácie na vizualizáciu dát z CPR a CALIOP buď nespĺňajú naše
požiadavky na vyextrahovanie maximálnej úrovne detailov a produkčnú kvalitu
výstupu, alebo vyžadujú veľmi nákladné platformy ako \program{ITTVIS IDL}, bolo
potrebné, aby sme vyvinuli vlastný nástroj. To sme si stanovili ako cieľ tejto práce.

Takýto nástroj sme vyvinuli a nazvali sme ho \ccplot. Je to
unixový program používajúci voľne dostupné knižnice a programovací jazyk
\program{Python}. Umožnuje vizualizáciu niekoľkých typov produktov nielen z CPR
a CALIOP, ale aj spektrorádiometra MODIS na satelite Aqua (tiež člen zostavy
A-Train), a tak užívateľom poskytuje pomerne široký prehľad o študovanej
situácii. \ccplot sme uverejnili prostredníctvom Internetu pod licenciou umožňujúcou
bezplatné používanie bez obmedzení, ako aj úpravu zdrojového kódu a redistribúciu.

V tejto práci sa sústredíme na predstavenie jednotlivých produktov,
vysvetlenie fyzikálneho princípu ich merania a spracovania, na technické detaily formátu
používaného na distribúciu produktov, detailný popis implementácie a algoritmov
použitých v programe \ccplot, a podanie návodu, ako používať tento program.
